\documentclass{beamer}
\usepackage{ctex}
%\usepackage{graphicx}
\usepackage{biblatex}
\usepackage{csquotes}

\addbibresource{main.bib}

\title{函数式编程与程序语言理论简介(内容尚不完整)}
\author{Shreck Ye}
%\institute{四川大学2015级}
\date{2023年12月}

\begin{document}

\maketitle



\begin{frame}{编程语言的常见范式}

    \cite{programming_paradigm_wikipedia}
    \begin{itemize}
        \item 命令式编程

              the programmer instructs the machine how to change its state

              包括常见的工业编程语言 C、C++、Java、JavaScript 等
              \begin{itemize}
                  \item 过程式编程
                  \item 面向对象编程
              \end{itemize}
        \item 函数式编程

              the desired result is declared as the value of a series of function applications
        \item 逻辑编程

              in which the desired result is declared as the answer to a question about a system of facts and rules
    \end{itemize}

\end{frame}

\begin{frame}{更具体地回答什么是函数式编程——一些显著特征}
    \begin{itemize}
        \item 不可变变量(由于不可变,“变量”在很多函数式语言中也叫做绑定(binding)或定义 (definition))
        \item 表达式代替语句
        \item 没有循环控制结构(for、while、goto 等)

              问题:怎么样实现循环?
    \end{itemize}

    如果一门语言上面三者的要求都严格满足,那么这门语言可以被称作“纯”函数式编程语言。

    根据以上内容很容易回答问题:为什么 C/C++、Java、Python、JavaScript 等一般不被称为函数式编程语言?
\end{frame}

\begin{frame}
    \frametitle{函数式编程的一些历史}
    \begin{itemize}
        \item Alonzo Church(阿隆佐·邱奇)提出(无类型)Lambda 演算法\cite{lambda_calculus_wikipedia}

              作者简介:函数式编程与程序语言理论的祖师爷、图灵在美国留学期间的导师

              这可以看作函数式编程的理论起源。

              Lambda calculus (also written as λ-calculus) is a formal system in mathematical logic for expressing computation based on function abstraction and application using variable binding and substitution.

              邱奇-图灵论题:可(有效)计算的函数可以 = 图灵机可计算的函数 = 邱奇的 Lambda 演算可计算的函数

        \item Lisp 语言

        \item Can programming be liberated from the von Neumann style?: a functional style and its algebra of programs

              作者 John Backus,简介:1977 年图灵奖得主、FORTRAN 语言的创造者

              这可以看作在工业界应用函数式语言的起源。
    \end{itemize}
\end{frame}

\begin{frame}
    \frametitle{函数式编程的一些概念}
    \begin{itemize}
        \item 头等函数(函数作为一等公民)
        \item 高阶函数
        \item Lambda表达式与闭包
        \item 代数数据类型(ADT)与模式匹配
    \end{itemize}
\end{frame}

\begin{frame}{其他的一些概念}
    %关注点分离

    %纯函数式编程的简洁定义(不可变)
    %Java可变的例子(见 Programming Languages, Part A)
    %与数学惯例类比:阶乘、gcd
    %递归(提一下尾递归)
    %对比凸显elegance

    %副作用的缺点例子->纯
\end{frame}


\begin{frame}{函数式编程在工业界的一些影响和应用}
    虽然上述众多工业界语言不是函数式语言,但都在近年来逐渐支持了函数式编程这一范式,最显著的一些特性便是 lambda 表达式和类型参数/泛型(Java)/模板(C++)的支持。

    常见的函数式编程语言有:Haskell、ML 系(Ocaml、Standard ML)、Scala、Lisp 语言及其各种变体、PureScript、F\#/F*

    集合高阶函数/集合函数式API/面向表达式编程

    问题:常见的一种编程需求是在一列元素里面找出需要的元素 -> filter
    \begin{itemize}
        \item map
        \item filter
        \item reduce
    \end{itemize}

    并行计算、分布式计算与大数据:Google MapReduce 与 Apache Spark

    一个数学原理:reduce 与结合律和群论

\end{frame}

\begin{frame}{一些常见的争议话题}
    一个大家比较认可的回答:
    \begin{itemize}
        \item 最好的编程语言是什么?
              现在来讲,没有一个单一的最好的编程语言,特别是对于工程

              普遍来讲,适合的就是最好的,适合体现在效率(开发效率与运行效率)、生态(第三方开源库生态与劳动力市场生态)等多方面

        \item 程序语言的语法很重要吗?

              语法的重要性并没有那么大,对语义的研究的重要性高于对语法的研究

              Wadler 定律(Wadler's Law)\cite{wadlers_law_haskellwiki}:

              \begin{displayquote}
                  In any language design, the total time spent discussing a feature in this list is proportional to two raised to the power of its position.
                  \begin{enumerate}
                      \setcounter{enumi}{0}
                      \item[0.] Semantics
                      \item[1.] Syntax
                      \item[2.] Lexical syntax
                      \item[3.] Lexical syntax of comments
                  \end{enumerate}
              \end{displayquote}
    \end{itemize}

\end{frame}

\begin{frame}
    \frametitle{总结和建议}
    \begin{itemize}
        \item 为什么要用函数式编程?

              优点:更为简洁、增加代码复用、让程序更容易分析、提高软件安全性(减少程序 bug)

              增加代码复用的例子:提取函数类型参数的公共函数


    \end{itemize}
\end{frame}

\begin{frame}{推荐学习材料}
    \begin{itemize}
        \item 学习你所熟悉的语言的函数式编程教程
        \item Coursera 课程
              \begin{itemize}
                  \item Programming Languages
                  \item Functional Programming in Scala
              \end{itemize}
        \item 知乎博主与其上的 Haskell 等语言的教程
    \end{itemize}
\end{frame}

\begin{frame}{推荐的一些编程语言(含个人见解)}
    希望学习函数式编程的特性与理论:
    \begin{itemize}
        \item[Haskell] 想学习函数式编程和其中的各种高级特性与体验惰性求值
        \item[ML] 同上但不想要惰性求值
        \item[Racket] Lisp 的现代版,动态类型函数式编程
    \end{itemize}

    将函数式编程应用到工程开发中:
    \begin{itemize}
        \item[Scala] 想要一门兼容多范式又简洁抽象的语言,能体验函数式编程的同时兼容面向对象编程与 Java 的生态(Spark、Kafka、Twitter 的实现语言,不过近年来发展比较颓势)
        \item[Kotlin] 简化版的 Scala,现在的最大特性和主要目标是接近平台原生性能的多平台开发(服务器端、web、Android、iOS)
        \item[TypeScript] 更加类型安全的 JavaScript
        \item[Rust] 剪掉了很多冗杂特性且更加类型安全、内存安全的同时不牺牲性能的面向未来的 C++
    \end{itemize}
\end{frame}

\begin{frame}
    \frametitle{更多}

    语法糖的含义与讨论

    可变性 + OOP 的封装在并发计算方面的缺陷

    函数式编程现今的一些局限性和缺点:在一些地方性能较差、学习难度更大
\end{frame}

% TODO 分成两个幻灯片

\begin{frame}
    \frametitle{依赖类型语言与辅助证明}

    Curry-Howard 对应:类型即命题,程序(实现)即证明
\end{frame}

\begin{frame}{依赖类型编程的一些实际应用}
    带维数的矩阵、参数约束(对比 guarded programming)
\end{frame}

\begin{frame}{Lambda 立方}
    值与类型的依赖关系
\end{frame}

\begin{frame}{数学归纳法的推广:结构归纳法}

    第二数学归纳法的推广:良基归纳法
\end{frame}

\begin{frame}
    \frametitle{与范畴论 的联系}

    currying 与范畴论

    Curry–Howard–Lambek 对应\cite{curry_howard_wikipedia}(Robert Harper 称为计算三位一体论(computational trinitarianism)):计算、逻辑、范畴/空间是看待数学基础的三种不同的角度。

\end{frame}

\begin{frame}{更多话题}
    \begin{itemize}
        \item[辅助证明与依赖类型语言] Coq、Agda、Lean、Idris、Epigram、Isabella 等等
        \item[MLTT 与 Coc]
        \item[同伦类型论(Homotopy Type Theory,简称 HoTT)与单价公理(Univalence Axiom)] 相等关系在依赖类型中被表示为想等类型,相等的证明被表示为相等类型的元素,那么相等类型的元素是否有多个?(以及它们与同伦群的关系)
            能不能把同构的类型看作相等的?
        \item[type case] 是否应该对类型进行模式匹配?
        \item[定理自动证明] 有了辅助证明语言以后,是否可以对数学定理进行自动证明?它有那些局限性?(哥德尔不完备定理)现在大火的基于机器学习的 AI 能不能帮上忙?
    \end{itemize}
\end{frame}


\begin{frame}
    程序语言理论与依赖类型相关的一些推荐内容:
    \begin{itemize}
        \item \href{https://infinity-type-cafe.github.io/ntype-cafe-summer-school/}{无穷类型咖啡(∞-type Café)暑假学校}
        \item \href{https://www.codewars.com/}{Codewars} 上有一些很不错的各方面入门练习题
    \end{itemize}
\end{frame}

\printbibliography

\end{document}
